\begin{table}[h]
    \caption{Summary statistics of the Taiwan Social Change Survey sample}
    \label{tab:summstat-tscs}

    \begin{tabular}{lccc}
        \toprule
        Variable                     & \mc{Obs.} & \mc{Mean} & \mc{Std. Dev.} \\ \midrule
        Self-reported son preference & 3,697     & 0.46      & 0.498          \\
        Gender quota proportion      & 3,697     & 0.145     & 0.0712         \\
        Woman                        & 3,697     & 0.498     & 0.5            \\
        Population size (thousand)   & 3,697     & 234.826   & 151.543        \\
        Years of education           & 3,697     & 10.6      & 4.68           \\
        Age                          & 3,697     & 45        & 16.3           \\ \bottomrule
    \end{tabular}

    \begin{tablenotes}
        ``Self-reported son preference'' is a dummy variable constructed using respondents' answers to a survey question asking about their own evaluation of ``the importance of having at least one son in order to continue the family bloodline''. The same question was asked in the 2001 and 2006 waves of TSCS, but the answers are recorded on different scales. They are on a scale of 4 in 2001, ranging over 1 ``very important'', 2 ``important'', 3 ``unimportant'', and 4 ``very unimportant''; and on a scale of 7 in 2006, ranging over 1 ``extremely important'', 2 ``very important'', 3 ``slightly important'', 4 ``neutral'', 5 ``slightly unimportant'', 6 ``very unimportant'', and 7 ``extremely unimportant''. To our end, we construct a dummy variable representing an answer of 1 or 2 in 2001, or any number from 1 to 3 in 2006, to indicate that the respondent attached at least some importance to having a son ($=1$) or not ($=0$).
    \end{tablenotes}
\end{table}