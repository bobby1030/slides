\begin{table}[]

    \caption{Estimating the effects of gender quota on household decisions making}
    \label{tab:household-decision-making}

    \resizebox{0.9\columnwidth}{!}{
        \begin{tabular}{l*{5}{d{0.3}}}
            \toprule
                                                    & \mc{(1)}          & \mc{(2)}       & \mc{(3)}           & \mc{(4)}       & \mc{(5)}             \\
                                                    & \mc{Expenditures} & \mc{Saving \&} & \mc{Allocation of} & \mc{Parenting} & \mc{1st prin. comp.} \\
                                                    & \mc{}             & \mc{finance}   & \mc{chores}        & \mc{}          & \mc{}                \\ \midrule
            Gender quota proportion                 & -0.085            & -0.123         & -0.135**           & -0.042         & -0.574*              \\
                                                    & (0.112)           & (0.086)        & (0.058)            & (0.083)        & (0.345)              \\
            Female $\times$ Gender quota proportion & 0.184***          & 0.234***       & 0.078*             & 0.030          & 0.669***             \\
                                                    & (0.069)           & (0.065)        & (0.047)            & (0.057)        & (0.258)              \\ \midrule
            Mean dep. var.                          & 0.833             & 0.895          & 0.938              & 0.882          & 0.158                \\
            Observations                            & \mc{17,358}       & \mc{17,013}    & \mc{17,358}        & \mc{16,384}    & \mc{16,039}          \\
            Adj. $R^2$                              & 0.015             & 0.042          & 0.00401            & 0.0556         & 0.0172               \\ \bottomrule
            \multicolumn{6}{l}{Standard errors are clustered at township-election level. * p<0.1, ** p<0.05, *** p<0.01.}                             \\
        \end{tabular}
    }
\end{table}